\-Collection of mex functions for \-Open\-C\-V library \begin{DoxyAuthor}{\-Author}
\-Kota \-Yamaguchi 
\end{DoxyAuthor}
\begin{DoxyDate}{\-Date}
\-March 2012
\end{DoxyDate}
\href{http://www.cs.stonybrook.edu/~kyamagu/mexopencv/}{\tt http\-://www.\-cs.\-stonybrook.\-edu/$\sim$kyamagu/mexopencv/}\hypertarget{index_development}{}\section{\-Developing a new mex function}\label{index_development}
\-All you need to do is to add your mex source file in {\ttfamily src/+cv/}. \-If you want to add a mex function called myfunc, create {\ttfamily src/+cv/myfunc}.cpp. \-The minimum contents of the myfunc.\-cpp would look like this\-:


\begin{DoxyCode}
    #include "mexopencv.hpp"
    void mexFunction( int nlhs, mxArray *plhs[],
                      int nrhs, const mxArray *prhs[] )
    {
        // Check arguments
        if (nlhs!=1 || nrhs!=1)
            mexErrMsgIdAndTxt("myfunc:invalidArgs","Wrong number of arguments")
      ;
        
        // Convert MxArray to cv::Mat
        cv::Mat mat = MxArray(prhs[0]).toMat();
        
        // Do whatever you want
        
        // Convert cv::Mat back to mxArray*
        plhs[0] = MxArray(mat);
    }
\end{DoxyCode}


\-This example simply copies an input to cv\-::\-Mat object and then copies again to the output. \-Notice how the {\ttfamily \hyperlink{class_mx_array}{\-Mx\-Array}} class provided by mexopencv converts mx\-Array to cv\-::\-Mat object. \-Of course you would want to do something more with the object. \-Once you create a file, type {\ttfamily cv.\-make} to build your new function. \-The compiled mex function will be located inside {\ttfamily +cv/} and accessible through {\ttfamily cv.\-myfunc} within matlab.

\-The {\ttfamily \hyperlink{mexopencv_8hpp}{mexopencv.\-hpp}} header includes a class {\ttfamily \hyperlink{class_mx_array}{\-Mx\-Array}} to manipulate {\ttfamily mx\-Array} object. \-Mostly this class is used to convert between opencv data types and {\ttfamily mx\-Array}.


\begin{DoxyCode}
    int i            = MxArray(prhs[0]).toInt();
    double d         = MxArray(prhs[0]).toDouble();
    bool b           = MxArray(prhs[0]).toBool();
    std::string s    = MxArray(prhs[0]).toString();
    cv::Mat mat      = MxArray(prhs[0]).toMat();   // For pixels
    cv::Mat ndmat    = MxArray(prhs[0]).toMatND(); // For N-D array
    cv::Point pt     = MxArray(prhs[0]).toPoint();
    cv::Size siz     = MxArray(prhs[0]).toSize();
    cv::Rect rct     = MxArray(prhs[0]).toRect();
    cv::Scalar sc    = MxArray(prhs[0]).toScalar();
    cv::SparseMat sp = MxArray(prhs[0]).toSparseMat(); // Only double to float
\end{DoxyCode}
 
\begin{DoxyCode}
    mxArray* plhs[0] = MxArray(i);
    mxArray* plhs[0] = MxArray(d);
    mxArray* plhs[0] = MxArray(b);
    mxArray* plhs[0] = MxArray(s);
    mxArray* plhs[0] = MxArray(mat);
    mxArray* plhs[0] = MxArray(ndmat);
    mxArray* plhs[0] = MxArray(pt);
    mxArray* plhs[0] = MxArray(siz);
    mxArray* plhs[0] = MxArray(rct);
    mxArray* plhs[0] = MxArray(sc);
    mxArray* plhs[0] = MxArray(sp); // Only 2D float to double
\end{DoxyCode}


\-Check {\ttfamily \-Mx\-Arraay.\-hpp} for the complete list of conversion \-A\-P\-I.

\-If you rather want to develop a matlab function that internally calls a mex function, make use of the {\ttfamily +cv/private} directory. \-Any function placed under private directory is only accessible from {\ttfamily +cv/} directory. \-So, for example, when you want to design a matlab class that wraps the various behavior of the mex function, define your class at {\ttfamily +cv/\-My\-Class}.m and develop a mex function dedicated for that class in {\ttfamily src/+cv/private/\-My\-Class\-\_\-}.cpp . \-Inside of {\ttfamily +cv/\-My\-Class}.m, you can call {\ttfamily \-My\-Class\-\_\-()} without cv namescope.\hypertarget{index_testing}{}\section{\-Testing}\label{index_testing}
\-Optionally, you can add a testing script for your new function. \-The testing convention in mexopencv is that testing scripts are all written as a static function in a matlab class. \-For example, {\ttfamily test/unit\-\_\-tests/\-Test\-Filter2\-D.\-m} is a class that describes test cases for filter2d function. \-Inside of the class, a couple of test cases are written as a static function whose name starts with 'test'.

\-If there is such a class inside {\ttfamily test/unit\-\_\-tests/} , typing `make test` would invoke all test cases and show your result. \-Use {\ttfamily test/} directory to place any resource file necessary for testing. \-An example of testing class is shown below\-:


\begin{DoxyCode}
    classdef TestMyFunc
        methods (Static)
            function test_1
                src = imread('/path/to/myimg');
                ref = [1,2,3];                  % reference output
                dst = cv.myfunc(src);           % execute your function
                assert(all(dst(:) == ref(:)));  % check the output
            end
            
            function test_error_1
                try
                    cv.myfunc('foo');           % myfunc should throw an error
                    error('UnitTest:Fail','myfunc incorrectly returned');
                catch e
                    assert(strcmp(e.identifier,'mexopencv:error'));
                end
            end
        end
    end
\end{DoxyCode}


\-In \-Windows, add path to the {\ttfamily test} directory and invoke {\ttfamily \-Unit\-Test} to run all the test routines.\hypertarget{index_documenting}{}\section{\-Documenting}\label{index_documenting}
\-You can create a \-Matlab help documentation for mex function by having the same file with '.m' extension. \-For example, on linux 64-\/bit architecture, the help file for {\ttfamily filter2\-D.\-mexa64} would be {\ttfamily filter2\-D.\-m}. \-Inside the help file should be only matlab comments. \-An example is shown below\-:


\begin{DoxyCode}
    %MYFUNC  brief description about myfunc
    %
    % Detailed description of function continues
    % ...
\end{DoxyCode}
 